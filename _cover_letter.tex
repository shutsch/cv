\begin{center}
\Huge{Statement of research interest}
\end{center}

\section{Summary and Motivation}

My past and current work focusses on large scale inference projects in connection to astrophysical magnetic fields, be it the three dimensional configuration of the field itself or connected observables.
To this end, I am employing versatile inference techniques based on Information field Theory (IFT), Nested Sampling and Machine Learning.
These techniques give me a rich toolbox to analyse astrophysical data and model , be it   \par
In the past, I have worked on cosmology, in particular on a prediction of the three dimensional primoridal magnetic field configuration in the Local Universe.
Furthermore, I have laid a strong focus on the Faraday effect as a tracer of the magnetizd ISM.
In this regard, I have lead a international collaboration of over 20 scientists to derive the most up to date map of the Galactic Faraday screen and demonstrated that this map contains surprising structure tracing   \par
Presently, I am leading the technical working group of the IMAGINE consortium to build a inference engine for the Galactic magnetic field.
I am also continuing my research on the Faraday sky by trying to disentangle the magnetic field and plasma information hidden by the
\par
My future research is driven by my interest in Galactic cartography, both from a methodologcal and a physical point of view.
In the former case, it is clear that the vast range of scale necessary of
In thA common theme in all my studies of the ISM is the close interconnection of all tracers.
It is amost impossible to study a single component of the ISM without either making strongly simplifiyng assumptions or ,
Thus makes it clear that the ultimate goal in Galactic cartography must be a holistic picture that connects all tracers and Galactic components.
For me personally this means, I have
I hope that the

\section{The Galactic Faraday sky}

The Faraday effect (i.e. the wavelength dependent rotation of linearly polarized light in a magnetized plasma) entangles information on the line-of-sight (LoS) component of the magnetic field with the thermal plasma density.
In can be probed at radio frequencies for both the diffuse ISM and
In the latter case, a

\begin{center}
  \includegraphics[width=0.7\textwidth]{RM_mean.png}
  \captionof{Figure}{\label{fig:faraday} \small\textit{The Faraday effect as induced by the Miky Way. This map was inferred form about 55000 extraglactic polarized point sources, mostly activ galactic nuclei.}}
\end{center}


\section{Inferring the Galactic magnetic field}

The Galactic magnetic field is traced by a plethora of effects, such as synchrotron radiation, the Faraday effect, dust polarisation in the optical and infrared, Zeeman splitting and more.
Surprisingly, albeit the existence of so many available observables probing a diverse range of Galactic environements, our knowledge on both the origin and the configuration of the Galactic magnetic field is still rather poor compared to other ISM components such as e.g. dust.
This is partly a result of the lack
This has motivated the formation IMAGINE consortium which aims at providing both a forum and
Within the consortium, I am responsible for the techical development of the IMAGINE software package, i.e. a inference engine that will connect all possible tracers and models with the aim
I also am currently supervising a Master student suing the IMAGINE software to

\section{Primordial magnetic fields}

In the beginning of my PhD, I have worked on a prediction of todays configuration of the primordial magnetic field induced by the Harrison effect, a battery effect that appears due to small perturations in the primordial plasma.
In that work, we have used the
While the resulting magnetic fields are to small to be observed by

\begin{center}
  \includegraphics[width=0.6\textwidth]{Primordial.png}
  \captionof{Figure}{\label{fig:primordial} \small\textit{The LoS averaged primordial magnetic field in Galactic coordinates, as induced by the Harrison effect.}}
\end{center}

\section{Miscellanous}
I furthermore contributed the to research on circular polarization of radio emission as a tracer of both the Galacitc ISM and  \par
Apart from my research on the Faraday sky and the Galactic magnetic field, I have contibuted to research in Oceanology, where I suoervised two Master students on bioluminescence in the Deep Sea.
We hhave reconstructed the location and movement of luminescent animals such as e.g. Plankton  \par
I also contributed to the technical development of the Nifty package,
